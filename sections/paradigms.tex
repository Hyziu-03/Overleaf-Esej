\newpage

\section{Paradygmaty programowania}

\subsection{Programowanie obiektowe}

\begin{flushleft}
    Programowanie obiektowe to jeden z paradygmatów programowania. Opiera się na wykorzystaniu obiektów odzwierciedlających to, co istnieje w rzeczywistości przypisując odpowiednie atrybuty i metody, aby umożliwić mu działanie. Obiekty mogą dziedziczyć  dane między sobą oraz \linebreak przekazywać je  sobie w inny sposób.
\end{flushleft}

\begin{flushleft}
    Pojęcie obiektu jest abstrakcyjne, co utrudnia zrozumienie tego stylu programowania osobom, które uczą się programować. Należy przez to rozumieć pojedynczy obszar programu, który przechowuje informacje na wybrany przez nas temat i może być wykorzystany wielokrotnie.
\end{flushleft}

\subsection{Podejście proceduralne}

\begin{flushleft}
    Programowanie proceduralne zostało stworzone z pomysłem, że programy są zbudowane z sekwencji instrukcji, które trzeba wykonać. To podejście skupia się przede wszystkim na podziale oprogramowania na ich zbiory zwane procedurami. \cite{procedure-deepsource} Mogą ona przechowywać dane lokalne, które nie są dostępne przez inne części programu.
\end{flushleft}

\begin{mdframed}[backgroundcolor=yellow!20]
    Wiele z najczęściej używanych współcześnie języków programowania \linebreak wspiera wiele jego paradygmatów \cite{oop-wikipedia}, dlatego możemy w kontrolowanym środowisku porównać, który sposób pisania oprogramowania jest bardziej optymalny, wydajny czy przyjemny dla osoby, która pisze kod.
\end{mdframed}
