\newpage

\section{Różnice w podejściach do programowania}

\subsection{Jak budujemy aplikacje?}

\begin{flushleft}
    Najważniejszą i najbardziej podstawową różnicą odnoszącą się do architektury programów jest szerokie spojrzenie. Paradygmat obiektowy nakłada na nas podejście \emph{bottom-up}, dzięki czemu skupiamy się na tym, jakie komponenty mają tworzyć ekosystem, w którym działa nasza aplikacja. 
\end{flushleft}

\begin{flushleft}
    Podejście proceduralne zakłada zgoła inne. Określamy je  \emph{top-down}. Jest ono dużo bardziej intuicyjne, ponieważ deklarujemy funkcje, które realizują ogólne zadania, dopiero potem je uszczegóławiając.
\end{flushleft}

\subsection{Zapewnianie integralności}

\begin{flushleft}
    Drugim ważnym aspektem jest stosunek danych do funkcji. Podejście obiektowe pozwala je łączyć, dzięki czemu unikamy błędów. Pisząc funkcje musielibyśmy samodzielne stworzyć mechanizmy, które zapewnią integralność danych, co dostajemy od razu wybierając program obiektowy.
\end{flushleft}

\subsection{Priorytetyzowanie bezpieczeństwa}

\begin{flushleft}
    Po trzecie definiując obiekty musimy od razu ustalić to, jakie części programy będą miały do nich dostęp, co zwiększa bezpieczeństwo. W ten sposób możemy zablokować nieprzewidziane działania w obrębie oprogramowania, które tworzymy. \cite{security}
\end{flushleft}

\begin{mdframed}[backgroundcolor=yellow!20]
    Współcześnie w wielu przypadkach spotkamy się z językami obiektowymi, takimi jak \emph{C++}, \emph{C\#}, \emph{Java} lub \emph{Python}. Języki stricte proceduralne są w wielu przypadkach przestarzałe, mowa o języku \emph{C}, \emph{BASIC}, \emph{COBOL} czy \emph{Pascal}. \cite{languages}
\end{mdframed}
