\newpage

\section{Programowanie obiektowe}

\subsection{Po co używać klas i obiektów?}

\begin{flushleft}
    Stworzenie projektu w paradygmacie obiektowym ułatwia jego rozwój w zespole. \cite{oop-benefits} Program może być podzielony na części realizujące oddzielne zadania.  Dodatkowymi zaletami tego stylu jest wielokrotne użycie tego samego kodu, dzięki czemu oszczędza się czas i nie powiela wykonanej wcześniej pracy.
 \end{flushleft}
 
\begin{flushleft}
    Co więcej ze względu na mechanizm \emph{enkapsulacji} złożony kod jest ukryty, a utrzymanie oprogramowanie staje się łatwiejsze i bardziej bezpieczne, co sprzyja szybciej realizacji zadań w zespole.
\end{flushleft} 

\subsection{Krytyka podejścia obiektowego}

\begin{flushleft}
    Profesor Uniwersytetu Harvarda Luca Andrea Cardelli twierdzi, że kod \linebreak napisany w paradygmacie obiektowym jest siłą rzeczy mniej efektywny niż kod proceduralny. \cite{oop-critics} 
\end{flushleft}
